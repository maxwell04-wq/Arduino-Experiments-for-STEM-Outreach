\chapter*{Preface}
\addcontentsline{toc}{chapter}{Preface}
\fancyhead[L]{Preface}

\begin{center}
\texturdu
{
اگر ہوعشق تو ہے کفر بھی مسلمانی\\
نہ ہو، تو مردِ مسلماں بھی کافر و زندیق\\\hspace{6cm}
(علامہ محمد اقبال ؒ)
}
\\[3mm]
{\small\textit{As good as Muslim’s true belief, If blessed with Love, the unfaith exist; Bereft of Love a Muslim true, Is no better than a heretic}}
\end{center}

\vspace{7mm}

It is a social responsibility to return back to the community one is a part of. However, after briefly getting involved in community service and while longly being stuck in a stagnant education system, I realised that my dissidence for obsolete science teaching is a convenient alternative of the altruistic motivation generally associated with humanitarian initiatives. Over the course of time, this self-centered search of refuge in STEM outreach only made me more empathetic towards the underprivileged students who have, quite frankly, had it worse than me.

 The key motivation of our outreach has been to present STEM as a lively discipline to the audience, allowing them to explore it beyond their textbooks. Therefore, we aimed at building experiments which were both aesthetically appealing and analytically evocative so as to entice the audience towards STEM. As the students of electrical engineering, we found the electronic experiments to be the most viable basis of our outreach, which not only allowed us to conveniently construct a diverse set of experiments from a limited amount of resources but also let us bring our penchant for engineering to the table.
 
 This book is the lovechild of our three-year-old journey of science outreach. We have archived in it all the experiments we have designed over the course of different outreach initiatives. The prime motivation to write this book was to preserve our work, no so much a narcissistic reminder of our own accomplishment as a guide to other aspiring volunteers who wish to embark on community outreach--altruistically or otherwise.   

The book focuses on two \textit{flavors} of outreach--as I like to call them: the first part of the book comprises of experiments built as a supplement for the science curriculum of primary schools; the second part consists of recreational projects which are built with the purpose of evoking a fascination for STEM among the audience while not being academically rigorous. While these flavors are distinct from each other and aim to communicate STEM in a different fashion, they are by no means the only flavors available. We hope the reader, after getting a taste of our efforts, is inspired to discover and explore other flavors.

This book encapsulates a fragment of our efforts for popularising STEM among the underprivileged students in our community. However, we strongly believe that it is merely a means to an end and, by no means, an end in itself. We hope that the reader finds inspiration in our work for their own ventures of STEM popularisation and STEM {humanisation}.

\vspace{2cm}
\begin{flushright}
{\large{Mahnoor Fatima}}
\end{flushright}